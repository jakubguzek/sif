\section{Kontrola jakości, rozpoznanie możliwości wdrożenia wybranych norm jakościowych}

Do głównych zadań kontroli jakości należy:
\begin{itemize}
	\item pobieranie prób
	\item przeprowadzanie badań -- materiałów wyjściowych oraz produktów gotowych
	\item badania stabilności produktów
	\item walidacja metod badawczych
\end{itemize}\vspace{\baselineskip}

W strukturze organizacyjnej wytwórni dział jakości zajmuje szczególne miejsce tzn. jest niezależny od każdej innej jednostki organizacyjnego. Posiada laboratorium z pracownikami i wyposażeniem umożliwiającym przeprowadzenie wszystkich niezbędnych kontroli i badań. Wszystkie dane i wyniki uzyskiwane podczas kontroli jakości muszą być wiarygodne, powtarzalne i oparte na solidnej wiedzy naukowej.\vspace{\baselineskip}

Laboratorium posiada procedury, instrukcje oraz określone specyfikacje dla wszystkich materiałów (produktów) oraz metody kontroli, aby mieć pewność, że:
\begin{itemize}
	\item osoby uprawnione posiadają wszystkie informacje konieczne, aby zdecydować czy dana seria produktu nadaje się do zwolnienia do obrotu
	\item drogą audytu możliwe jest odtworzenie historii każdej wytworzonej serii
\end{itemize}\vspace{\baselineskip}

W opracowanych dokumentach prowadzi się kontrolę wprowadzanych zmian, które są udokumentowane i uzasadnione. Archiwizowane są oryginału dokumentów, a stosowane nadzorowane kopie. Czas i miejsce przechowywania archiwalnych oryginałów jest określone; miejsce przechowywania odpowiednio zabezpieczone, aby zapewnić odpowiedni stan i czytelność archiwizowanych dokumentów.

Sprzęt wykorzystywany w laboratoriach kontroli jakości jest kalibrowany i sprawdzany odpowiednimi metodami, w określonych odstępach czasu; urządzenia niesprawne, w miarę możliwości usunięte lub wyraźne oznakowane jako nieprawne. Jeżeli wyposażenie nie podlega wzorcowaniu jest wyraźnie oznakowane w sposób odróżniające je od tego, które wymaga wzorcowania. Oznakowanie powinno pozwolić na stwierdzenie, kiedy ma być przeprowadzone następne wzorcowanie.

Ocena każdej i szczególnej serii produktu jest oparta na badaniach, które zostały przeprowadzone na reprezentacyjnej próbce. Próbka jest reprezentatywna, jeżeli dostarcza informacji całej serii materiału lub produktu.\vspace{\baselineskip}

Jeżeli próbka ma myc reprezentatywna to powinna zostać pobrana:
\begin{itemize}
	\item z materiału, który jest jednorodny
	\item w odpowiedni sposób
	\item za pomocą odpowiedniego sprzętu
	\item do odpowiednich pojemników
	\item w odpowiedniej tłuści
	\item w odpowiednim pomieszczeniu
	\item przez upoważnionego pracownika
	\item z zachowaniem należytych środków ostrożności i zasad BHP
\end{itemize}\vspace{\baselineskip}

Pobrane próbki do bieżących badań analitycznych i mikrobiologicznych oraz próbki archiwalne:
\begin{itemize}
	\item wyrobów gotowych, w opakowaniach bezpośrednich, w ilościach wystarczających na wykonanie 1-2 pełnych analiz zgodnych ze specyfikacją
	\item materiałów wyjściowych
\end{itemize}\vspace{\baselineskip}

Na podstawie wyników badań reprezentatywnej próbki upoważniona osoba podejmuje decyzje o zwolnieniu danej partii materiału do produkcji wyrobu gotowego do obrotu. Przy czym istotne są dwie zasady:
\begin{enumerate}
	\item Można używać tylko materiałów wyjściowych i opakowaniowych zwolnionych przez Dział Kontroli Jakości oraz będących w okresie ważności
	\item Żadna seria produktu nie może zostać zwolniona do obrotu zanim osoba uprawniona nie wyrazi pisemnej zgody potwierdzającej zgodność serii z wymaganiami określonymi w specyfikacji.
\end{enumerate}\vspace{\baselineskip}

Wszystkie podejrzane wyniki, które są niezgodne z wymaganiami specyfikacji lub ustalonymi kryteriami akaparacji określa się jako wyniki poza specyfikacją (OOS) i muszą być one wyjaśnione w toku specjalnego postępowania Działu Kontroli Jakości. Wynik OOS nie zawsze oznacza wadliwą serię i konieczność jej odrzucenia. Ważne jest jednak, aby każdy wynik OOS został zaszeregowany do jednego z 3 możliwych kategorii błędów:
\begin{itemize}
	\item błędu laboratoryjnego
	\item błędu produkcyjnego niezależnego od procesu
	\item błędu procesu produkcyjnego
\end{itemize}\vspace{\baselineskip}

Przebieg działań wyjaśniających wynik OOS musi być udokumentowany i szczegółowo opisany w specjalnym raporcie.
