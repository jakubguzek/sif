\section{Program i sezonowość produkcji, warianty produkcji w sezonie letnim i zimowym, bilans materiałowy}

Projekt zakłada pracę 5 dni w tygodniu w godzinach 6-22. przez cały rok (z wyłączeniem świąt). Produkowane będą 3 typy odżywek: proteinowa, emolientowa, humektantowa i 2 typy szamponów: ziołowy oraz peelingujący. Wszystkie z tych produktów wytwarzane będą z jednakową intensywnością niezależnie od pory roku. Każdego dnia w zakładzie będzie przebiegać jeden cykl produkcyjny dla każdego typu produktu.

\begin{table}[h]
	\centering
	\caption{Program produkcji}
	\begin{tabular}{lllll}
		\hline
		\textbf{Produkt} & \makecell[l]{\textbf{Produkcja w} \\ \textbf{skali roku [kg]}} & \makecell[l]{\textbf{Produkcja w} \\ \textbf{skali roku [hl]}} & \makecell[l]{\textbf{Produkcja na} \\ \textbf{dobę [kg]}} & \makecell[l]{\textbf{Produkcja na} \\  \textbf{dobę [hl]}} \\
		\hline\hline
	Odżywka proteinowa & 150\,000 & 1\,500 & 625,2 & 6,252 \\
	Odżywka emolientowa & 150\,000 & 1\,500 & 625,2 & 6,252 \\
	Odżywka humektantowa & 150\,000 & 1\,500 & 625,2 & 6,252 \\
	Szampon ziołowy & 330\,000 & 3\,300 & 1\,375 & 13,75 \\
	Szampon wzbogacony & 330\,000 & 3\,300 & 1\,375 & 13,75 \\
	\hline
	\end{tabular}
\end{table}

\begin{table}[h]
	\centering
	\caption{Zapotrzebowanie na surowce do produkcji szampon}
	\begin{tabular}{llll}
		\hline
		\textbf{Produkt} & \textbf{Surowiec} & \makecell[l]{\textbf{Zapotrzebowanie} \\ \textbf{dobowe [kg]}} & \makecell[l]{\textbf{Zapotrzebowanie} \\ \textbf{roczne [kg]}} \\
		\hline\hline
		\multirow{12}{*}{Szampon ziołowy} & Woda & 1090,375 & 261690 \\
		 & Sodium Coco Sulfate & 137,5 & 33000 \\
		 & Rubus idaeus L. Leaf Extract & 110 & 26400 \\
		 & Urtica dioica L. Leaf Extract & 96,25 & 23100 \\
		 & Coco-Glucoside & 71,5 & 17160 \\
		 & Cocamidopropyl Betaine & 55 & 13200 \\
		 & Thymus vulgaris L Leaf Extract & 45,375 & 10890 \\
		 & Sodium Benzoate & 34,375 & 8250 \\
		 & Guar Hydroxypropyltrimonium Chloride & 27,5 & 6600 \\
		 & Citric Acid & 24,75 & 5940 \\
		 & Sodium Chloride & 13,75 & 3300 \\
		 & Tocopherol & 2,75 & 660 \\
		\hline
		\multirow{12}{*}{Szampon peelingujący} & Woda & 660 & 158400 \\
		 & Coffea arabica L. Bean Grind & 204,875 & 49170 \\
		 & Sodium Coco Sulfate & 185,625 & 44550 \\
		 & Glycerin & 68,75 & 16500 \\
		 & Prunus Amygdalus Dulcis (Sweet Almond) Oil & 55 & 13200 \\
		 & Cocamidopropyl Betaine & 55 & 13200 \\
		 & Coco-Glucoside & 55 & 13200 \\
		 & Citric Acid & 48,125 & 11550 \\
		 & Guar Hydroxypropyltrimonium Chloride & 26,125 & 6270 \\
		 & Sodium Benzoate & 15,125 & 3630 \\
		 & Sodium Chloride & 13,75 & 3300 \\
		 & Tocopherol & 1,375 & 330 \\
		 \hline
	\end{tabular}
\end{table}


\begin{table}[h]
	\centering
	\caption{Zapotrzebowanie na surowce do produkcji odżywek}
	\begin{tabular}{llll}
		\hline
		\textbf{Produkt} & \textbf{Surowiec} & \makecell[l]{\textbf{Zapotrzebowanie} \\ \textbf{dobowe [kg]}} & \makecell[l]{\textbf{Zapotrzebowanie} \\ \textbf{roczne [kg]}} \\
		\hline\hline
		\multirow{14}{*}{Odżywka proteinowa} & Woda & 306,348 & 73500 \\
		 & Cetearyl Alkohol & 137,544 & 33000 \\
		 & Quercus robur L. Bark Extract & 43,764 & 10500 \\
		 & Glycerin & 31,26 & 7500 \\
		 & Wheat Amino Acids & 28,134 & 6750 \\
		 & Behentrimonium Chloride & 25,008 & "6000 \\
		 & Polyglyceryl-3 PCA & 12,504 & 3000 \\
		 & Potassium Sorbate & 12,504 & 3000 \\
		 & Sodium Benzoate & 12,504 & 3000 \\
		 & Dehydroacetic Acid & 6,252 & 1500 \\
		 & Citric Acid & 5,6268 & 1350 \\
		 & Lactic Acid & 1,2504 & 300 \\
		 & Linalool & 1,2504 & 300 \\
		 & Tetrasodium EDTA & 1,2504 & 300 \\
		\hline
		\multirow{14}{*}{Odżywka emolientowa} & Woda & 250,08 & 60000 \\
		 & Butyrospermum Parkii (Shea Butter) & 87,528 & 21000 \\
		 & Vitis labrusca fruit extract & 59,394 & 14250 \\
		 & Argan Oil & 53,142 & 12750 \\
		 & Black Cumin Oil & 50,016 & 12000 \\
		 & Cetearyl Alcohol & 43,764 & 10500 \\
		 & Macadamia Ternifolia Seed Oil & 25,008 & 6000 \\
		 & Brassica Oleracea Italica Seed Oil & 18,756 & 4500 \\
		 & Prunus Domestica (Plum) Seed Oil & 18,756 & 4500 \\
		 & Camellia Japonica Seed Oil & 12,504 & 3000 \\
		 & Benzoic Acid & 1,8756 & 450 \\
		 & Dehydroacetic Acid & 1,8756 & 450 \\
		 & Phenoxyethanol & 1,2504 & 300 \\
		 & Linalool & 1,2504 & 300 \\
		\hline
		\multirow{9}{*}{Odżywka humektantowa} & Woda & 287,592 & 69000 \\
		 & Aloe Barbadensis (Aloe Vera) Leaf Juice & 125,04  & 30000 \\
		 & Cetearyl Alcohol & 62,52 & 15000 \\
		 & Glycerin & 56,268 & 13500 \\
		 & Persea Gratissima Oil & 43,764 & 10500 \\
		 & Behentrimonium Chloride & 31,26 & 7500 \\
		 & Polyglyceryl-3 PCA & 12,504 & 3000 \\
		 & Benzoic Acid & 3,126 & 750 \\
		 & Dehydroacetic Acid & 3,126 & 750 \\
		 \hline
	\end{tabular}
\end{table}

\textbf{Przykładowe obliczenia:}

Woda (odżywka proteinowa):

\begin{equation}
	0,147 \frac{kg}{butelka} \cdot 2\,084 \frac{butelek}{doba} = 306.348kg; \qquad
	0,147 \frac{kg}{butelka} \cdot 500\,000 \frac{butelek}{rok} = 73\,500
\end{equation}

Tocopherol (szampon ziołowy):
\begin{equation}
	1 \frac{g}{butelka} \cdot 2\,750 \frac{butelek}{doba} = 2.75 kg; \qquad
	1 \frac{g}{butelka} \cdot 660\,000 \frac{butelek}{rok} = 660kg
\end{equation}

