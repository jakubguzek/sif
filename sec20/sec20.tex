\section{Rodzaje odpadów i ich dobowa ilość}

Razem około 15tys. litrów ścieków na dobę. Odprowadzane do systemu kanalizacji.
\begin{itemize}
\item Ścieki technologiczne – woda zużyta do mycia aparatury i urządzeń oraz sprzątania pomieszczeń - około 3l na każdy litr produktu, czyli 14 tys. litrów na dobę
\item Ścieki bytowo-gospodarcze – produkowane przez pracowników - ok. 15l na jednego pracownika, czyli 1tys. litrów na dobę
\end{itemize}\vspace{\baselineskip}

Odpady są segregowane. Zakład ma podpisaną umowę z lokalną firmą zajmującą się wywozem i utylizacją śmieci. Odpady odbierane są z zakładu w każdy poniedziałek, środę oraz piątek.
\begin{itemize}
\item Opakowania po komponentach na kosmetyki – kartony, beczki, worki, palety
\item Wadliwe opakowania na kosmetyki
\item Odpady powstające na hali produkcyjnej
\item Odpady komunalne - związane z bytowaniem pracowników
\end{itemize}\vspace{\baselineskip}

Zakład ma podpisaną umowę z firmą zajmującą się utylizacją odpadów potencjalnie niebezpiecznych dla środowiska. Odpady odbierane są z zakładu w każdy poniedziałek.
\begin{itemize}
\item Próbki kosmetyków poddawane analizie w laboratorium
\item Odczynniki chemiczne używane do analiz oraz płytki z pożywkami i wyhodowanym na nich materiałem biologicznym
\item Opakowania po odczynnikach chemicznych
\item Wadliwe partie produktów
\end{itemize}
