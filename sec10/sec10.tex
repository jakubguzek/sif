\section{Normy i akty prawne}

\begin{itemize}
	\item Prawo budowlane
	\item Prawo ochrony środowiska
	\item Prawo energetyczne
	\item USTAWA o planowaniu i zagospodarowaniu przestrzennym
	\item USTAWA o efektywności energetycznej
	\item Rozporządzenie w sprawie ogólnych przepisów bezpieczeństwa i higieny pracy
	\item Prawo wodne
	\item Dyrektywa w sprawie monitorowania temperatur w środkach transport
	\item Kosmetyk naturalny musi być wyprodukowany zgodnie ze standardem GMP (Good Manufacturing Practice – Dobra Praktyka Produkcyjna). Standard GMP określa norma ISO 22716:2007, która precyzuje wymagania dotyczące m.in. personelu, pomieszczeń czy urządzeń stosowanych do produkcji kosmetyków.
	\item Produkt, który ma zostać wprowadzony na rynek, należy notyfikować w systemie Cosmetic Products Notification Portal (CPNP), bazie zbierającej informacje na temat kosmetyków wprowadzonych do obrotu na terenie Unii Europejskiej.
	\item Rozporządzenie 1223/2009, dot. informacji na opakowaniu produktu kosmetycznego
	\item Rozporządzenie 655/2013 dot. deklaracji marketingowych
	\item Norma 16128-1:2016 oraz 16128-2:2017 określa definicje naturalnych i organicznych składników oraz prezentuje metodologię obliczania indeksów naturalności, naturalnego pochodzenia oraz organiczności i organicznego pochodzenia.
	\item Rozporządzenie 1223/2009 wymaga, aby wszystkie informacje dotyczące bezpieczeństwa zostały zawarte w raporcie bezpieczeństwa produktu kosmetycznego.
	\item Rozporządzenie 655/2013/ WE [2, 9].  ustawa opakowaniowa
	\item Pozwolenie wodno – prawne
	\item Ustawa z dnia 4 października 2018 r. o produktach kosmetycznych
	\item Ustawa z dnia 30 marca 2001 r. o kosmetykach
	\item Rozporządzenie Ministra Zdrowia z dnia 20 lutego 2019 r. w sprawie Ośrodka administrującego Systemem Informowania o Ciężkich Działaniach Niepożądanych Spowodowanych Stosowaniem Produktów Kosmetycznych.
	\item Rozporządzenie Ministra Zdrowia z dnia 28 lutego 2019 r. w sprawie określenia wzorów wniosków oraz zaświadczeń związanych z wykazem zakładów wytwarzających produkty kosmetyczne.
	\item Rozporządzenie Ministra Zdrowia z dnia 28 lutego 2019 r. w sprawie ośrodka uprawnionego do dostępu do informacji o produkcie kosmetycznym.
	\item Rozporządzenie Ministra Zdrowia z dnia 19 marca 2020 r. w sprawie metod oznaczeń próbek niezbędnych do kontroli bezpieczeństwa produktów kosmetycznych.
	\item Rozporządzenie Ministra Opieki Społecznej z dnia 18 stycznia 1939 r. s prawie  dozoru nad wyrobem i obiegiem środków kosmetycznych.
	\item Zarządzenie Ministrów: Przemysłu Rolnego i Spożywczego, Przemysłu Drobnego i Rzemiosła, Handlu Wewnętrznego oraz Zdrowia w sprawie unormowania produkcji i dystrybucji wyrobów kosmetycznych.
	\item Dyrektywa Komisji 95/17/WE z dnia 19 czerwca 1995 r. ustanawiająca szczegółowe zasady stosowania dyrektywy Rady 76/768/EWG w odniesieniu do nieumieszczania jednego lub kilku składników w wykazie używanym do etykietowania produktów kosmetycznych.
	\item Siódma Dyrektywa Komisji 96/45/WE z dnia 2 lipca 1996 r. odnosząca się do metod analizy niezbędnych do kontroli składu produktów kosmetycznych.
	\item k.p.a. – ustawa z dnia 14 czerwca 1960 r. – Kodeks postępowania administracyjnego (t. jedn. Dz. U. 2018, poz. 2096)
	\item u.i.h. - ustawa z dnia 15 grudnia 2000 r. o Inspekcji Handlowej (t. jedn. Dz. U. 2018, poz. 1930 z późn. zm.)
	\item u.k. – ustawa z dnia 30 marca 2001 r. o kosmetykach (t. jedn. Dz. U. 2013, Nr 475 z późn. zm.)
	\item u.b.p. - ustawa z dnia 12 grudnia 2003 r. o ogólnym bezpieczeństwie produktów (t. jedn. Dz.U. 2016, poz. 2047) 41
	\item u.o.k.k. – ustawa z dnia 16 lutego 2007 roku o ochronie konkurencji i konsumentów (t. jedn. Dz.U.2018, poz.798) u.p.k. – ustawa z dnia 4 października 2018 r. o produktach kosmetycznych (Dz.U. 2018, poz. 2227)
	\item u.p.n.p.r. – ustawa z dnia 23 sierpnia 2007 roku o przeciwdziałaniu nieuczciwym praktykom rynkowym (t. jedn. Dz.U.2017, poz.2070)
	\item u.p.p. – ustawa z dnia 26 stycznia 1984 roku – Prawo prasowe (t. jedn. Dz.U.2018, poz.1914)
	\item u.z.n.k. – ustawa z dnia 16 kwietnia 1993 roku o zwalczaniu nieuczciwej konkurencji (t. jedn. Dz.U.2018, poz.419)
	\item RODO - Rozporządzenie Parlamentu Europejskiego i Rady (UE) 2016/679 z dnia 27 kwietnia 2016 r. w sprawie ochrony osób fizycznych w związku z przetwarzaniem danych osobowych i w sprawie swobodnego przepływu takich danych oraz uchylenia dyrektywy 95/46/WE (ogólne rozporządzenie o ochronie danych); Dz. Urz. UE L 119 z 04.05.2016, s. 1.
	\item Rozporządzenie 1223/2009 – Rozporządzenie Parlamentu Europejskiego i Rady z dnia 30 listopada 2009 r. dotyczące produktów kosmetycznych (Dz. Urz. L 342 z 22.12.2009, s. 59-209)
	\item Decyzja wykonawcza Komisji z dnia 25 listopada 2013 r. w sprawie wytycznych dotyczących załącznika I do rozporządzenia Parlamentu Europejskiego i Rady (WE) nr 1223/2009 dotyczącego produktów kosmetycznych (Dz.U. L z 2013, poz. 315, s. 82—104)
	\item Podręcznik Grupy Roboczej ds. Produktów Kosmetycznych (Podgrupy ds. produktów z pogranicza) dotyczący stosowania Rozporządzenia Parlamentu Europejskiego i Rady (WE) nr 1223/2009 z dnia 30 listopada 2009 r. dotyczącego produktów kosmetycznych (wersja 1.0 listopad 2013 r.)
	\item Rozporządzenie 655/2013/WE określające wspólne kryteria dotyczące uzasadniania oświadczeń stosowanych w związku z produktami kosmetycznymi oraz wytyczne do tego rozporządzenia
	\item Sprawozdanie Komisji Dla Parlamentu Europejskiego i Rady z dnia 19.9.2016 r. na temat oświadczeń o produktach sporządzanych na podstawie wspólnych kryteriów w branży kosmetycznej COM/2016/0580 final
	\item Rozporządzenie Ministra Zdrowia z dnia 16 czerwca 2003 r. w sprawie określenia kategorii produktów będących kosmetykami (Dz.U. z 2003 poz. 125, Nr 1168)
	\item Kodeks Etyki Reklamy opracowany przez Związek Stowarzyszeń Rady Reklamy,
	\item Zasady Przewodnie Stowarzyszenia Cosmetics Europe w zakresie Reklamy i Oświadczeń Marketingowych,
	\item Karta Odpowiedzialnej Reklamy i Komunikacji Marketingowej Stowarzyszenia Cosmetics Europe.
\end{itemize}
