\section{Warunki magazynowania surowców, produktów i niektórych materiałów pomocniczych}

Ze względów funkcjonalnych jak i higienicznych, magazyn główny firmy \textsf{Sif} będzie podzielony na trzy części:
\begin{itemize}
	\item magazyn surowców do produkcji;
	\item magazyn gotowego produktu;
	\item magazyn pustych opakowań
\end{itemize}

Ponadto funkcję magazynu będzie pełniła część powierzchni hali produkcyjnej, na której będą znajdowały się paletopojemniki z wodą, a także część przeznaczona na przechowywanie środków czystości.

Magazyn będzie znajdował się niedaleko hali produkcyjnej, co zapewni dogodny dostęp do serowca podczas przygotowywania produktu oraz jego sprawne zmagazynowanie po pokończonym procesie.

Założenia dotyczące magazynowania:
\begin{itemize}
	\item Magazyn będzie wyposażony w regały paletowe rzędowe pozwalające na składowanie towarów spaltyzowanych i niespaletyzowanych. Pojedynczy regł ma długość 6.6\,m, szerokość 2\,m oraz 2 (dwa) poziomy. Pozwala na składowanie palet po obydwu stronach w ilości 10 sztuk na jednym poziomie czyli 20 sztuk na wszystkich poziomach łącznie
	\item Wysokość magazynu -- 6\,m
	\item 70\% magazynu stanowi pojedyncza powierzchnia składowań, 30\% powierzchnia strefy buforowej przy rampach, przeznaczona do tymczasowego składowania rozładowywanych surowców lub produktów przygotowywanych do wysyłki
	\item Korytarze pomiędzy regałami, a także ścianą i regałami będą wynosiły około 4\,
\end{itemize}\vspace{\baselineskip}

\textbf{Magazyn na surowce}

Obliczenia dotyczące liczby jednorazowo składowanych europalet (na przykładzie firmy \textsf{TRB Natural Extraxt}):

Dostarczanych będzie 170 sztuk worków o pojemności 25\,kg jednego surowca oraz 77 sztuk drugiego. Na jednej europalecie może być składowane 35 worków o pojemność 25\,kg
\[
	5\,\mathrm{szt.} \cdot 7\,\text{ poziomów}
\]
Zatem łącznie dostarczanych będzie 7 europalet
\[
	\frac{170}{35} + \frac{77}{35} = 7
\]
Liczba europalet dostarczanych przez pozostaję firmy została wyznaczona w analogiczny sposób i wynosi dla firmy:
\begin{itemize}
	\item \textsf{OQEMA} -- 33 sztuk europalet/paletopojemników
	\item \textsf{Distripark} -- 10 sztuk europalet
	\item \textsf{CIECH S.A.} -- 1 sztuka europalety
	\item \textsf{Sigma-Aldrich} -- 4 sztuki europalet
	\item \textsf{PK Components} -- 10 sztuk europalet
	\item \textsf{Supreme Gums} -- 2 sztuki europalet
\end{itemize}

Liczba jednorazowo składowanych europalet w magazynie surowców wynosi 67.\vspace{\baselineskip}

Obliczenia dotyczące powierzchni magazynów:
\begin{itemize}
	\item Pojedynczy regał ma miejsca na 20 palet, a magazyn musi pomieścić 67 palet
	\item Zakładamy, że 10\% lokalizacji zostaję wolnych, aby zapewnić płynność pracy magazynu w przypadku spiętrzeń przepływu towaru, zatem liczba palet mieszczących się w 1 regale wynosi 18 ($20 \cdot 0.9$)
	\item Magazym musi zawierać 4 regały (67/18 = 3.7 = 4)
	\item Pomiędzy ścianami a regałem oraz pomiędzy regałami są korytarze 4\,m, jest 5 korytarzy wzdłuż ścian krótszych -- szerokość, oraz 2 wzdłuż ścian dłuższych -- długość.
	\item Regały ustawione są wzdłuż ścian krótszych (szerokość)
	\item Szerokość regału to 2\,m, a długość to 6.6\,m
	\item Długość magazynu to $5 \cdot 4\mathrm{m} + 4 \cdot 2\mathrm{m} = 28\mathrm{m}$
	\item Szerokość magazynu to $2\cdot 4\mathrm{m} + 6.6\mathrm{m} = 15\mathrm{m}$
	\item Powierzchnia składowania to $15\mathrm{m} \cdot 28\mathrm{m} = 420\mathrm{m}$
	\item Powierzchnia składowania stanowi 70\% zatem doliczając powierzchnię strefy buforowej uzyskujemy łączną powierzchnię magazynu wynoszącą $\mathbf{600\mathrm{m^{3}}}$
\end{itemize}\vspace{\baselineskip}

\textbf{Magazyn pustych opakowań}:
\begin{itemize}
	\item Musi pomieścić miesięczny zapas opakowań oraz etykiet, czyli 147 palet
	\item Do tego celu posłuży 8 regałów
	\item Długość magazynu -- 44\,m
	\item Szerokość magazynu -- 15\,m
	\item Powierzchnia składowania -- 660\,$\mathrm{m^{2}}$
	\item Powierzchnia składowani stanowi 70\% zatem doliczając powierzchnię strefy buforowej uzyskujemy łączną powierzchnię magazynu wynoszącą $\mathbf{945m^{2}}$ (długość 63\,m)
\end{itemize}\vspace{\baselineskip}

\textbf{Nagazyn gotowego produktu}:
\begin{itemize}
	\item Musi pomieścić tygodniową produkcję szamponów i odżywek, czyli 26 palet.
	\item Do tego celu posłużą 2 regały.
	\item Długość magazynu: 12',m
	\item Szerokość magazynu: 15',m
	\item Powierzchnia składowania: 180$\mathrm{m^{2}}$
	\item Powierzchnia składowania stanowi 70\% zatem doliczając powierzchnię strefy buforowej uzyskujemy łącznbie powierzchnię magazynu wynoszącą $\mathbf{260m^{2}}$ (długość 17\,m).
\end{itemize}

\begin{table}
	\centering
	\caption{Zestawienie powierzchni}
	\begin{tabular}{cc}
		\hline
		\textbf{Magazyn} & \textbf{Powierzchnia [$\mathbf{m^{3}}$]} \\
		\hline\hline
		Surowców & 600 \\
		Pustych opakowań & 945 \\
		Gotowych produktów & 260 \\
		\textbf{Łącznie} & 1805 \\
		\hline
	\end{tabular}
\end{table}

