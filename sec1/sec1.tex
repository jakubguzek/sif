\section{Charakterystyka programowo-technologiczna}

Firma kosmetyczna \textsf{Sif} skupia swoją całą uwagę na produkcji kosmetyków do włosów, wyłącznie naturalnych opartych na równowadze PEH i metodzie mycia OMO. Zarówno siedziba firmy kosmetycznej, jak i linia produkcyjna znajdują się w miejscowości Sowia Wola. Jest ona położona w województwie mazowieckim, w powiecie nowodworskim, w gminie Czosnów. Nieduża odległość od miast st. Warszawy (39km), a także bliskie położenie względem Wisłostrady, zapewnia idealny tunel komunikacyjny. Lokalizacja zapewnia możliwość sprawnej dystrybucji produktu. Docelowa produkcja zakładu będzie wynosiła 1\,110 ton kosmetyków.
Produkty dedykowane są osobom, które stosują świadomą pielęgnację włosów lub chcą ją dopiero zacząć. \vspace{\baselineskip}

W dotychczas stworzonej ofercie figurują produkty takie jak.:

\begin{itemize}
	\item 3 Odżywki do włosów:
	\begin{itemize}
	\item Proteinowa;
	\item Emolientowa;
	\item Humektantowa.
	\end{itemize}
	\item 2 Szampony myjące do włosów
	\begin{itemize}
	\item Lekki szampon ziołowy;
	\item Szampon wzbogacony o mocniejszy detergent oraz składniki o właściwościach peelingujących.
	\end{itemize}
\end{itemize}\vspace{\baselineskip}

\textbf{Opisy produktów:}
\begin{itemize}
	\item Odżywka proteinowa -- planowana roczna produkcja wynosi 150 ton (lub 500 tys. butelek o pojemności 300ml). Lekka formuła zbudowana na bazie mleka owsianego i dodatkiem startej kory dębu Nadaje przyjemny zapach. Odżywka wypełnia mikrouszkodzenia, nadaje włosom naturalną objętość i blask.
	\item Odżywka emolientowa -- planowana produkcja roczna wynosi 150 ton (lub 500 tys. butelek o pojemności 300ml). Odżywka do włosów o gęstej formule stworzona z oleju arganowego i oleju z czarnuszki, tworzy zwartą powłokę zabezpieczającą strukturę włosa przez czynnikami mechanicznymi, a dzięki kremowej formule, włosy stają się gładkie i miękkie.
	\item Odżywka humektantowa -- planowna roczna produkcja w wynosi 150 ton (lub 500 tys. butelek o pojemności 300ml). Połączenie olejków z awokado i aloesu, zapewnia właściwe nawilżenie i tym samym zdrowy wygląd włosów.
	\item Lekki szampon ziołowy -- roczna produkcja planowana na 330 ton (lub 660 tys. butelek o pojemności 500ml). Pozbawiony mocnych detergentów, składowo uzupełniony o zioła takie jak liście maliny właściwej, pokrzywy zwyczajnej i tymianku.
	\item Szampon wzbogacony -- planowana roczna produkcja wynosi 330 ton (lub 660 tys. butelek o pojemności 500ml) Zawiera mocniejszy detergent oraz składniki o właściwościach peelingujących, w tym elementy peelingu ze zmielonych ziaren kawy.
\end{itemize}\vspace{\baselineskip}

\textbf{Wszystkie kosmetyki wyróżniają się:}
\begin{itemize}
\item wyłącznie naturalnymi składnikami;
\item składem opartym na równowadze PEH (proteinowo-emolientowo-humektantowej);
\item tym, że są przyjazne dla środowiska (zarówno pod względem składników kosmetyków, jak i opakowań z plastiku biodegradowalnego, we współpracy z zewnętrzną polską firmą ekologiczną);
\item certyfikatem i potwierdzeniem „cruelty free” tj. nietestowaniem produktów na zwierzętach na żadnym etapie produkcji i po jej zakończeniu;
\item wegańskimi składnikami wykluczającymi składniki takie jak: tłuszcze czy białka pochodzenia zwierzęcego.
\end{itemize}\vspace{\baselineskip}

Do produkcji produktów zawierających w sobie zioła używane są ekstrakty ziołowe pochodzenia naturalnego. Ekstrakt z liści maliny używany jest w ilości 26,4 tony w skali roku, a ekstrakt z pokrzywy właściwej w ilości 23,1 tony w skali roku. Ekstrakt z tymianku używany jest w ilości 10,89 tony w skali roku.

Szampon zawierający kawę, zawiera ją w postaci drobno zmielonych ziaren, które używane są w ilości 49,17 tony w skali roku.\vspace{\baselineskip}

Używana w zakładzie woda spełnia standardy sanitarne, a na etapach produkcji najbardziej wrażliwych i narzucających konieczność wysokiej sterylności używana jest woda oczyszczana w systemie Mili-Q.
