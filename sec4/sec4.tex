\section{Opis proponowanych do realizacji procesów technologicznych}

Lekki szampon ziołowy oraz szampon wzbogacony będą produkowane na jednej linii, która będzie myta między produkcją jednego i drugiego szamponu. Produkcja obu szamponów będzie wynosiła po 660\,tys. butelek o pojemności 500\,ml rocznie, co daje 2\,750 butelek dziennie (1\,375\,l) każdego szamponu przy pracy linii 5 dni w tygodniu w godzinach 6-22. Linia technologiczna będzie składała się z:
\begin{itemize}
	\item Zbiornik na wodę destylowaną typu  paletopojemnik (mauzer) wraz z pompą \textsf{Speroni CAM80}  
	\item Mieszalnik o pojemności 1\,500\,l
	\item Pompa dozującą \textsf{EURALCA}
	\item Maszyna do napełniania płynów, zakręcania butelek i etykietowania
\end{itemize}\vspace{\baselineskip}

\begin{enumerate}
	\item Pracownicy pompują wodę destylowaną oraz dodają według opracowanej formuły pozostałe składniki danego szamponu do mieszalnika, tak aby zrobić porcję o wielkości 1\,380\,l. Surowce są dostarczane wózkami widłowymi do miejsca produkcji. Pracownicy  włączają mieszalnik i ustawiają odpowiednie parametry:
	\begin{itemize}
		\item temperatura – 25\degree C,
		\item szybkość mieszania – 960 $obr/min$,
		\item czas – 2,5\,h.
	\end{itemize}
	\item Po zakończeniu procesu mieszania pracownicy pobierają 50ml próbki mieszaniny, przenoszą ją do laboratorium, a następnie badają czy spełnia wszystkie wymagane normy. Ocenie podlegają pH, gęstość, zapach i wygląd szamponu, a także zawartość detergentu.
	\item Jeśli szampon pozytywnie przejdzie test jakości to jest pompowany z mieszalnika do maszyny dozującej go w ilości po 500\,ml do jednocześnie 16 butelek.
	\item Butelki przechodzą do następnej maszyny, która je zakręca.
	\item Kolejna maszyna etykietuje butelki. 
	\item Pracownicy pakują gotowe produkty do pudełek w ilości po 16 sztuk; pudełka składują na paletach, a następnie zabierają je do dystrybutorni, skąd odbierane będą raz w tygodniu.
	\item Pracownicy myją cały sprzęt i przygotowują go do produkcji kolejnego szamponu. 
\end{enumerate}\vspace{\baselineskip}

Wszystkie odżywki będą produkowane na drugiej linii. Proces produkcji będzie wyglądał analogicznie jak w przypadki szamponów, z tym że:
\begin{itemize}
	\item Produkcja każdego rodzaju odżywki będzie wynosiła po 500\,tys. butelek o pojemności 300\,ml rocznie, co będzie dawało 2\,084 butelek dziennie (625,2\,l) każdego szamponu.
	\item Dzienna porcja pojedynczej odżywki w mieszalniku będzie wynosiła 630\,l.
	\item Maszyna dozująca dozuje odżywki w ilości 300\,ml do 16 butelek jednocześnie.
\end{itemize}
