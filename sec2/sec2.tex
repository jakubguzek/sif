\section{Baza surowcowa -- charakterystyka surowców (odmiana, jakość surowców, formy oceny jakości)}

\subsection{Składy poszczególnych produktów}

\begin{itemize}
\item odżywka proteinowa
\item odżywka emolientowa
\item odżywka humektantowa
\item szampon ziołowy
\item szampon peelingujący
\end{itemize}\vspace{\baselineskip}

\begin{table}[H]
\centering
	\caption{(a) i (b) składy poszczególnych szamponów. (c), (d) i (e) składy poszczególnych odżywek}
\begin{footnotesize}
	\begin{subfigure}[t]{0.45\textwidth}
		\centering
		\caption{Szampon ziołowy}
		\begin{tabular}{p{0.6\textwidth}ll}
			\hline
			\multirow{2}{*}{\textbf{Składniki}} & \multicolumn{2}{l}{\textbf{Ilość}} \\
			\cline{2-3}
			& \textbf{Procent} & \textbf{Masa} \\
			\hline\hline
			Aqua (Woda) & 54.50 & 396.5g  \\
			Sodium Coco Sulfate & 10.00 & 50g  \\
			Rubus idaeus L. Leaf Extract & 8.00 & 40g  \\
			Urtica dioica L. Leaf Extract & 7.00 & 35g  \\
			Coco-Glucoside & 5.20 & 26g  \\
			Cocamidopropyl Betaine & 4.00 & 20g  \\
			Thymus vulgaris L Leaf Extract & 3.30 & 16.5g  \\
			Sodium Benzoate & 2.50 & 12.5g  \\
			Guar Hydroxypropyltrimonium Chloride & 2.00 & 10g  \\
			Citric Acid & 1.80 & 9g  \\
			Sodium Chloride & 1.50 & 5g  \\
			Tocopherol & 0.20 & 1g  \\
			\hline
		\end{tabular}
	\end{subfigure}
	\begin{subfigure}[t]{0.5\textwidth}
		\centering
		\caption{Szampon peelingujący}
		\begin{tabular}{p{0.6\textwidth}lll}
			\hline
			\multirow{2}{*}{\textbf{Składniki}} & \multicolumn{2}{l}{\textbf{Ilość}} \\
			\cline{2-3}
			& \textbf{Procent} & \textbf{Masa}  \\
			\hline\hline
			Aqua (Woda) & 48.00 & 240g  \\
			Coffea arabica L. Bean Grind & 14.90 & 74.5g  \\
			Sodium Coco Sulfate & 13.50 & 67.5g  \\
			Glycerin & 5.00 & 25g  \\
			Prunus Amygdalus Dulcis (Sweet Almond) Oil & 4.00 & 20g  \\
			Cocamidopropyl Betaine & 4.00 & 20g  \\
			Coco-Glucoside & 4.00 & 20g  \\
			Citric Acid & 2.50 & 17.5g  \\
			Guar Hydroxypropyltrimonium Chloride & 1.90 & 9.5g  \\
			Sodium Benzoate & 1.10 & 5.5g  \\
			Sodium Chloride & 1.00 & 5g \\
			Tocopherol & 0.10 & 0.5g  \\
			\hline
		\end{tabular}
	\end{subfigure}

	\begin{subfigure}[t]{0.45\textwidth}
		\centering
		\caption{Odżywka proteinowa}
		\begin{tabular}{p{0.6\textwidth}lll}
			\hline
			\multirow{2}{*}{\textbf{Składniki}} & \multicolumn{2}{l}{\textbf{Ilość}} \\
			\cline{2-3}
			& \textbf{Procent} & \textbf{Masa} \\
			\hline\hline
			Aqua (Woda)  & 49.00 & 147g \\
			Cetearyl Alkohol  & 22.00 & 66g \\
			Quercus robur L. Bark Extract   & 7.00 & 21g \\
			Glycerin  & 5.00 & 15g \\
			Wheat Amino Acids   & 4.50 & 13.5g \\
			Behentrimonium Chloride  & 4.00 & 12g \\
			Polyglyceryl-3 PCA  & 2.00 & 6g \\
			Potassium Sorbate  & 2.00 & 6g \\
			Sodium Benzoate  & 2.00 & 6g \\
			Dehydroacetic Acid  & 1.00 & 3g \\
			Citric Acid  & 0.90 & 2.7g \\
			Lactic Acid  & 0.20 & 0.6g \\
			Linalool & 0.20 & 0.6g \\
			Tetrasodium EDTA  & 0.20 & 0.6g \\
			\hline
		\end{tabular}
	\end{subfigure}
	\begin{subfigure}[t]{0.5\textwidth}
		\centering
		\caption{Odżywka emolientowa}
		\begin{tabular}{p{0.6\textwidth}ll}
			\hline
			\multirow{2}{*}{\textbf{Składniki}} & \multicolumn{2}{l}{\textbf{Ilość}} \\
			\cline{2-3}
			& \textbf{Procent} & \textbf{Masa} \\
				\hline\hline
				Aqua (Woda) & 40.00 & 120g \\
				Butyrospermum Parkii (Shea Butter) & 14.00 & 42g \\
				Vitis labrusca fruit extract & 9.50 & 28.5g \\
				Argan Oil & 8.50 & 25.5g \\
				Black Cumin Oil & 8.00 & 24g \\
				Cetearyl Alcohol & 7.00 & 21g \\
				Macadamia Ternifolia Seed Oil & 4.00 & 12g \\
				Brassica Oleracea Italica Seed Oil & 3.00 & 9g \\
				Prunus Domestica (Plum) Seed Oil & 3.00 & 9g \\
				Camellia Japonica Seed Oil & 2.00 & 6g \\
				Benzoic Acid & 0.30 & 0.9g \\
				Dehydroacetic Acid & 0.30 & 0.9g \\
				Phenoxyethanol & 0.20 & 0.6g \\
				Linalool & 0.20 & 0.6g \\
			\hline
		\end{tabular}
	\end{subfigure}

	\begin{subfigure}{0.5\textwidth}
		\centering
		\caption{Odżywka humektantowa}
		\begin{tabular}{p{0.6\textwidth}ll}
			\hline
			\multirow{2}{*}{\textbf{Składniki}} & \multicolumn{2}{l}{\textbf{Ilość}} \\
			\cline{2-3}
			& \textbf{Procent} & \textbf{Masa} \\
				\hline\hline
				Aqua (Woda) & 46.00 & 138g \\
				Aloe Barbadensis (Aloe Vera) Leaf Juice & 20.00 & 60g \\
				Cetearyl Alcohol & 10.00 & 30g \\
				Glycerin  & 9.00 & 27g \\
				Persea Gratissima Oil & 7.00 & 21g \\
				Behentrimonium Chloride & 5.00 & 15g \\
				Polyglyceryl-3 PCA & 2.00 & 6g \\
				Benzoic Acid & 0.50 & 1.5g \\
				Dehydroacetic Acid & 0.50 & 1.5g \\
				\hline
			\end{tabular}
	\end{subfigure}
\end{footnotesize}
\end{table}


\begin{table}[H]
\centering
\caption{Planowane roczne zużycie surowców do produkcji:}
\begin{footnotesize}
	\begin{subfigure}{0.7\textwidth}
		\centering
		\caption{szamponów}
		\begin{tabular}{p{0.60\textwidth}ll}
			\hline
			\textbf{Składnik} & \makecell[l]{\textbf{Planowane} \\ \textbf{roczne zużycie}} & \textbf{Uwagi} \\
			\hline\hline
			Aqua (Woda) & 420\,090\,kg & \\
			Sodium Coco Sulfate & 77\,550\,kg & \\
			Coco-Glucoside & 60\,720\,kg & \\
			Coffea arabica L. Bean Grind & 49\,170\,kg & \\
			Sodium Benzoate & 43\,680\,kg & \\
			Rubus idaeus L. Leaf Extract & 26\,400\,kg & \\
			Cocamidopropyl Betaine & 26\,400\,kg & \\
			Urtica dioica L. Leaf Extract & 23\,100\,kg & \\
			Citric Acid & 17\,490\,kg & \\
			Glycerin & 16\,500\,kg & \\
			Prunus Amygdalus Dulcis (Sweet Al- mond) Oil & 13\,200\,kg & \\
			Guar Hydroxypropyltrimonium Chlo- ride & 12\,870\,kg & \\
			Thymus vulgaris L Leaf Extract & 10\,890\,kg & \\
			Sodium Chloride & 6\,600\,kg & \\
			Tocopherol & 990\,kg & \\
			\hline
		\end{tabular}
	\end{subfigure}

	\begin{subfigure}{0.7\textwidth}
		\centering
		\caption{odżywek}
		\begin{tabular}{p{0.60\textwidth}ll}
			\hline
			\textbf{Składnik} & \makecell[l]{\textbf{Planowane} \\ \textbf{roczne zużycie}} & \textbf{Uwagi} \\
			\hline\hline
			Aqua (Woda) & 202\,500\,kg & \\
			Cetearyl Alkohol & 58\,500\,kg & \\
			Aloe Barbadensis (Aloe Vera) Leaf Juice & 30\,000\,kg & \\
			Butyrospermum Parkii (Shea Butter) & 21\,000\,kg & \\
			Glycerin & 21\,000\,kg & \\
			Vitis labrusca fruit extract & 14\,250\,kg & \\
			Behentrimonium Chloride & 13\,500\,kg & \\
			Argan oil & 12\,750\,kg & \\
			Black cumin oil & 12\,000\,kg & \\
			Quercus robur L. Bark Extract & 10\,500\,kg & \\
			Persea Gratissima Oil & 10\,500\,kg & \\
			Wheat Amino Acids & 6\,750\,kg & \\
			Macadamia Ternifolia Seed Oil & 6\,000\,kg & \\
			Polyglyceryl-3 PCA & 6\,000\,kg & \\
			Brassica Oleracea Italica Seed Oil & 4\,500\,kg & \\
			Prunus Domestica (Plum) Seed Oil & 4\,500\,kg & \\
			Camellia Japonica Seed Oil & 3\,000\,kg & \\
			Potassium Sorbate & 3\,000\,kg & \\
			Sodium Benzoate & 3\,000\,kg & \\
			Dehydroacetic Acid & 2\,700\,kg & \\
			Citric Acid & 1\,500\,kg & \\
			Benzoic Acid & 1\,200\,kg & \\
			Linalool & 1\,200\,kg & \\
			Lactic Acid & 450\,kg & \\
			Tetrasodium EDTA & 300\,kg & \\
			Phenoxyethanol & 300\,kg & \\
			\hline
		\end{tabular}
	\end{subfigure}
\end{footnotesize}
\end{table}

\subsection{Charakterystyka poszczególnych składników produktów}

\textbf{Składniki szamponów:}

\begin{itemize}
\item \textbf{Sodium Coco Sulfate}

Sodium Coco Sulfate to sól sodowa siarczanu alkoholi tłuszczowych z oleju kokosowego. Jest półsyntetyczną substancją myjącą pochodzenia roślinnego, a konkretnie - anionowym związkiem powierzchniowo-czynnym. W kosmetykach pełni funkcję detergentu bądź emulgatora. Posiada silne właściwości oczyszczające, pianotwórcze (jest odpowiedzialna za wytwarzanie i stabilizację piany) i antystatyczne (pomaga zapobiec np. nadmiernemu elektryzowaniu włosów). SCS otrzymuje się przy pomocy kwasu siarkowego (VI) lub tlenku siarki (VI), które reagują z alkoholami tłuszczowymi, a następnie zostają zobojętnione. Sodium Coco Sulfate posiada mniejsze potencjalnie drażniące właściwości niż szeroko stosowane w kosmetyce detergenty SLS i SLES, wciąż jednak stanowi dość silny środek myjący.

\item \textbf{Coco-Glucoside}

Poliglukozyd kwasów oleju kokosowego. Niejonowa substancja powierzchniowo czynna. Substancja hydrofilowa, bardzo dobrze rozpuszczalna w wodzie. Bezpieczna dla środowiska - biodegradowalna. Stabilna chemicznie. Niewrażliwa na zmiany pH. Substancja bardzo łagodna dla skóry i błon śluzowych. Łagodzi ewentualne działanie drażniące wywołane przez anionowe substancje powierzchniowo czynne. Substancja myjąca - usuwa zanieczyszczenia z powierzchni skóry i włosów. Emulgator O/W, składnik umożliwiający powstanie emulsji. Substancja pianotwórcza, stabilizująca i poprawiająca jakość piany w mieszaninie z anionowymi substancjami powierzchniowo czynnymi. Pełni rolę modyfikatora reologii (czyli poprawia konsystencję) w preparatach myjących, zawierających anionowe substancje powierzchniowo czynne, dzięki tworzeniu tzw. mieszanych miceli. Ponadto pełni rolę solubilizatora, czyli umożliwia wprowadzanie do roztworu wodnego substancji nierozpuszczalnych lub trudno rozpuszczalnych w wodzie, np. kompozycje zapachowe, wyciągi roślinne, substancje tłuszczowe.

\item \textbf{Glycerin}

Gliceryna, alkohol trójwodorotlenowy. Hydrofilowa substancja nawilżająca. Ma zdolność przenikania przez warstwę rogową naskórka, dzięki czemu pełni rolę promotor przenikania - ułatwia w ten sposób transport innych substancji w głąb skóry. Humektant - zapobiega krystalizacji (wysychaniu) masy kosmetycznej przy ujściu butelki, tuby itp. Wspomaga działanie konserwujące poprzez obniżenie aktywności wody, która jest doskonałą pożywką dla drobnoustrojów.

\item \textbf{Cocamidopropyl Betaine}

Kokamidopropylobetaina. Amfoteryczna substancja powierzchniowo czynna. Substancja myjąca - usuwa zanieczyszczenia z powierzchni skóry i włosów. Substancja bardzo łagodna dla skóry i błon śluzowych, łagodzi ewentualne działanie drażniące anionowych substancja powierzchniowo czynnych. Substancja pianotwórcza, stabilizująca i poprawiająca jakość piany w mieszaninie z anionowymi substancjami powierzchniowo czynnymi. Pełni rolę modyfikatora reologii (czyli poprawia konsystencję) w preparatach myjących, zawierających anionowe substancje powierzchniowo czynne, dzięki tworzeniu tzw. mieszanych miceli.

\item \textbf{Polyglyceryl-3 PCA}

Ester kwasu stearynowego i trójpoliglicerolu. Niejonowa substancja powierzchniowo czynna pochodzenia roślinnego lub zwierzęcego, z estrów polimeru gliceryny i tłuszczowego kwasu stearynowego. Nierozpuszczalna w wodzie. Emolient tłusty, tworzy ochronny i natłuszczający film na powierzchni skóry i włosów, który kondycjonuje, zmiękcza i wygładza powierzchnię. Ma również właściwości emulgatora do tworzenia emulsji typu

O/W, zapobiega rozwarstwianiu faz, jest substancją pianotwórczą i myjącą oraz zdolną do tworzenia miceli w produktach z anionowymi surfaktantami. Podwyższa lepkość.

\item \textbf{Prunus Amygdalus Dulcis (Sweet Almond) Oil}

Olej ze słodkich migdałów ma bursztynowy kolor oraz lekki charakterystyczny zapach. W temperaturze pokojowej jest ciekły. Olej ze słodkich migdałów zawiera kwas oleinowy, linolowy oraz witaminy: A, B1, B2, B6, D i E. Tłoczony z nasion drzewa migdałowego. Emolient tzw. tłusty. Jeśli jest stosowany na skórę w stanie czystym, może być komedogenny, czyli sprzyjać powstawaniu zaskórników. Zastosowany w preparatach do pielęgnacji skóry i włosów tworzy na powierzchni warstwę okluzyjną (film), która zapobiega nadmiernemu odparowywaniu wody z powierzchni (jest to pośrednie działanie nawilżające), przez co kondycjonuje, czyli zmiękcza i wygładza skórę i włosy.

%\item \textbf{Glyceryl Oleate}

%Monogliceryd kwasu oleinowego. Niejonowa substancja powierzchniowo czynna. Nierozpuszczalna w wodzie. Stabilna w kwaśnym pH, niestabilna w silnie alkalicznym. Kompatybilna z twardą wodą, zawierającą głównie jony wapnia i magnezu. Emolient tzw. tłusty - stosowany w formie nierozcieńczonej może powodować powstawanie zaskórników. Zastosowany w preparatach do pielęgnacji skóry i włosów tworzy na ich powierzchni warstwę okluzyjną (film), która zapobiega nadmiernemu odparowywaniu wody z powierzchni (jest to pośrednie działanie nawilżające), przez co kondycjonuje skórę i włosy. Powstały film, wygładza powierzchnię naskórka i włosów. Nadaje połysk. Substancja zwilżająca - ułatwia usuwanie zanieczyszczeń. W preparatach myjących stosowany jako substancja renatłuszczająca.Proces mycia powoduje usunięcie m.in. substancji tłuszczowych, dlatego stosuje się substancje renatłuszczające, które odbudowują barierę lipidową. Emulgator W/O, składnik umożliwiający powstanie emulsji.

\item \textbf{Citric Acid}

Kwas cytrynowy należy do grupy alfahydroksykwasów (AHA). Substancja rozpuszczalna w wodzie. Substancja należy do alfahydroksykwasów (AHA) wykazuje działanie keratolityczne, czyli złuszczające, dzięki czemu usuwa przebarwienia i rozjaśnia skórę. Pełni rolę sekwestranta, czyli substancji, która kompeksuje jony metali, dzięki czemu zwiększa trwałość kosmetyku oraz jego stabilność. Regulator pH.

\item \textbf{Guar Hydroxypropyltrimonium Chloride}

Antystatyk tworzący na włosach ochronną powłokę, dodatkowo ma również właściwości odżywcze, wygładza, wzmacnia włosy, ułatwia rozczesywanie.

%\item \textbf{Hydrogenated Palm Glycerides Citrate}

%Utwardzone (Uwodornione) glicerydy palmy olejowej z kwaskiem cytrynowym. Ester kwasku cytrynowego w uwodornionych poprzez przyłączenie atomów wodoru glicerydach/tłuszczach palmy olejowej. Substancja słabo rozpuszczalna w wodzie. Emulgator do tworzenia emulsji typu O/W oraz substancja myjąca. Ma także właściwości kondycjonujące.

\item \textbf{Sodium Chloride}

Nieorganiczny związek chemiczny. Występuje w postaci bezbarwnych kryształów. Dobrze rozpuszcza się w wodzie. Modyfikator reologii. Wpływa na konsystencję kosmetyków myjących - powoduje wzrost lepkość w preparatach zawierających anionowe substancje powierzchniowo czynne.

\item \textbf{Potassium Sorbate}

Sorbinian potasu. Sól kwasu karboksylowego. Występuje w postaci białego, krystalicznego proszku. Substancja konserwująca, która uniemożliwia rozwój i przetrwanie mikroorganizmów w czasie przechowywania produktu. Chroni również kosmetyk przed wtórnym zakażeniem mikroorganizmami które możemy wprowadzić przy codziennym użytkowaniu produktu. Składnik dozwolony do stosowania w kosmetykach w ograniczonym stężeniu. Znajduje się na liście substancji konserwujących dozwolonych do stosowania z ograniczeniami w produktach kosmetycznych. Jego dopuszczalne maksymalne stężenie w gotowym produkcie to 0,6\% w przeliczeniu na kwas sorbowy.

\item \textbf{Sodium Benzoate}

Sól kwasu benzoesowego. Substancja konserwująca, która uniemożliwia rozwój i przetrwanie mikroorganizmów w czasie przechowywania produktu. Chroni również kosmetyk przed nadkażeniem bakteryjnym, które możemy wprowadzić przy codziennym użytkowaniu produktu. Składnik dozwolony do stosowania w kosmetykach w

ograniczonym stężeniu. Znajduje się na liście substancji konserwujących dozwolonych do stosowania z ograniczeniami w produktach kosmetycznych Jego dopuszczalne maksymalne stężenie to 0,5\% (w przypadku stosowania soli tego kwasu jest to 0,5\% w przeliczeniu na czysty kwas benzoesowy).

\item \textbf{Tocopherol}

Witamina E. Organiczny związek chemiczny zawierający w swojej strukturze pierścień 6-chromanolu, do którego dołączony jest izoprenowy łańcuch boczny. Nierozpuszczalny w wodzie. Rozpuszczalny w węglowodorach, alkoholach, tłuszczach i olejach. Określenie "witamina E" odnosi się do grupy związków naturalnych, w skład której wchodzą tokoferole i tokotrienole. Najwyższą aktywność biologiczą wykazuje D-$\alpha$-tokoferol, który jest głównym składnikiem naturalnej witaminy E. Substancja o działaniu antyoksydacyjnym (przeciwutleniającym), hamuje procesy starzenia się skóry wywoływane np. promieniowaniem UV lub dymem papierosowym. Jest doskonałym czynnikiem hamującym rodnikowe utlenianie lipidów w naskórku i skórze właściwej. Witamina E wykazuje zdolność wbudowywania się w struktury lipidowe błon komórkowych i cementu międzykomórkowego warstwy rogowej, dzięki czemu wzmacnia barierę naskórkową. Wzmocnienie bariery naskórkowej nie tylko utrudnia wnikanie substancji obcych i zapobiega podrażnieniom, ale także hamuje TEWL (transepidermalną utratę wody) dzięki czemu wpływa na poprawę nawilżenia skóry. Zapobiega powstawaniu stanów zapalnych, wzmacnia ściany naczyń krwionośnych i poprawia ukrwienie skóry. Przeciwutleniacz (antyoksydant). Zapobiega lub w znaczny sposób ogranicza szybkość zachodzenia procesu utleniania zawartych w kosmetyku składników tłuszczowych, np. niektórych cennych olejów roślinnych. Dodatek antyoksydantów zapewnia trwałość produktów, wydłuża ich przydatność do użycia, zabezpiecza przed powstawaniem nieprzyjemnego zapachu, zmianami barwy oraz konsystencji produktu gotowego.

\item \textbf{Linalool}

Nienasycony alkohol alifatyczny z grupy terpenów. Imituje zapach konwalii. Znajduje się na liście potencjalnych alergenów.

%\item \textbf{Limonene}

%Jednopierścieniowy weglowodór terpenowy. Imituje zapach skórki cytrynowej. Znajduje się na liście potencjalnych alergenów.

%\item \textbf{Geraniol}

%Nienasycony alkohol alifatyczny z grupy terpenów. Imituje zapach pelargonii. Znajduje się na liście potencjalnych alergenów.
\end{itemize}

\textbf{Składniki odżywek:}

\begin{itemize}
\item \textbf{Macadamia Ternifolia Seed Oil}

Olej z nasion makadamia. Jasnożółty olej otrzymywany poprzez tłoczenie nasion makadamii. Zawiera 57\% kwasu olejowego, 25\% kwasu palmitynowego, 15\% nasyconych kwasów tłuszczowych, bogaty w witaminy: A, B, E oraz składniki mineralne. Emolient tzw. tłusty. Jeśli jest stosowany na skórę w stanie czystym, może być komedogenny, czyli sprzyjać powstawaniu zaskórników. Zastosowany w preparatach do pielęgnacji skóry i włosów tworzy na ich powierzchni warstwę okluzyjną (film), która zapobiega nadmiernemu odparowywaniu wody z powierzchni (jest to pośrednie działanie nawilżające), przez co kondycjonuje skórę i włosy. Powstały film, wygładza powierzchnię naskórka i włosów.

\item \textbf{Cetearyl Alcohol}

Mieszanina alkoholu cetylowego i stearylowego. Niejonowa substancja powierzchniowo czynna. Alkohol cetylowy i alkohol stearylowy są głównymi składnikami tworzącymi alkohol cetylostearylowy. Należy do alkoholi tłuszczowych. Ma konsystencję stałego wosku. Emolient tzw. tłusty. Jeśli jest stosowany na skórę w stanie czystym, może być komedogenny, czyli sprzyjać powstawaniu zaskórników. Zastosowany w preparatach do pielęgnacji skóry i włosów tworzy na powierzchni warstwę okluzyjną (film), która zapobiega nadmiernemu odparowywaniu wody z powierzchni (jest to pośrednie działanie nawilżające), przez co kondycjonuje, czyli zmiękcza i wygładza skórę i włosy. Substancja konsystencjotwórcza, wpływa na lepkość gotowego produktu i poprawia właściwości użytkowe (aplikacyjne). Stabilizator emulsji.

\item \textbf{Behentrimonium Chloride}

Chlorek dokozylotrójmetyloamoniowy.. Czwartorzędowa sól amoniowa. Substancja łatwopalna, żółta, woskopodobna, pochodna tłuszczowego alkoholu behenowego pochodzenia roślinnego lub z wosku pszczelego. Rozpuszczalna w gorącej wodzie lub tłuszczach. Dopuszczalne stężenie w kosmetykach 0,1\%. Substancja bardzo często stosowana w produktach do mycia oraz pielęgnacji włosów, szamponach, odżywkach, maskach. Ma działanie myjące, zapobiega elektryzowaniu i plątaniu włosów, ułatwia rozczesywanie, wygładza, zmiękcza i kondycjonuje. Wykazuje również lekkie działanie konserwujące.

\item \textbf{Brassica Oleracea Italica Seed Oil}

Olejek pozyskiwany z nasion brokułu. Szczególnie nadaje się do pielęgnacji włosów, nadaje włosom połysk, może być użyty jako zastępstwo silikonów. On ma silne działanie kondycjonowania i wygładzania, łagodzi skręcanie się włosów.

\item \textbf{Moringa Oleifera Seed Oil}

Olej z nasion Moringi Olejodajnej. Olej bogaty w kwas oleinowy. Emolient, tworzy warstwę okluzyjną (film) na powierzchni skóry i włosów, która zapobiega nadmiernemu odparowywaniu wody z powierzchni (pośrednie działanie nawilżające). Działa kondycjonująco, nawilżająco i regenerująco.

\item \textbf{Prunus Domestica (Plum) Seed Oil}

Olej z pestek śliwki/olej śliwkowy. Ma działanie antyoksydacyjne i przeciwzapalne. Zawiera kwas oleinowy (do 80\%), nie pozostawia lepkiej warstwy i bardzo szybko się wchłania. Idealnie nadaje się do masażu ciała i twarzy. Działa kojąco i nawilżająco, a zawarte w nim fitosterole wpływają na poprawę jędrności skóry, wykazując działanie przeciwzmarszczkowe. Jako dodatek w mieszankach olejowych przedłuża trwałość delikatnych, podatnych na psucie olejów, zmniejsza wrażenie tłustości, przyspiesza wchłanianie oraz wzmacnia przenikanie składników aktywnych do głębszych warstw skóry. Olej śliwkowy jest również bogaty w witaminę E, która pełni funkcję antyoksydantu, zwalczającego wolne rodniki, dzięki temu olej chroni strukturę skóry przed zniszczeniami powodowanymi czynnikami zewnętrznymi.

\item \textbf{Camellia Japonica Seed Oil}

Olejek Tsubaki. Uzyskany z cennej odmiany kamelii, posiada niezwykłe właściwości odżywcze, zmiękczające i regenerujące, zapobiega przedwczesnemu starzeniu się włosów; nadaje matowym włosom połysk i witalność oraz sprawia, że stają się mocniejsze i grubsze.

%\item \textbf{Starch Hydroxypropyltrimonium Chloride}

%Emolinet, regulator lepkości, substancja antysatyczna. W szamponach i odżywkach do włosów, sprawia, że włosy są łatwe do rozczesania, elastyczne, miękkie i lśniące.

%\item \textbf{Cetrimonium Chloride}

%Chlorek cetylotrójmetyloamoniowy Nazwa zwyczajowa: Czwartrzędowa sól amoniowa. Kationowa substancja powierzchniowo czynna. Dzięki temu, że substancja posiada ładunek dodatni z łatwością łączy się z ujemnie naładowaną powierzchnią włosa. Substancja kondycjonująca włosy: poprawia rozczesywalność, zapobiega splątywaniu, nadaje połysk i wygładza włosy, wykazuje działanie antystatyczne, dzięki czemu zapobiega elektryzowaniu się włosów. Ułatwia spłukiwanie preparatu. Cetrimonium Chloride również obniża napięcie międzyfazowe, co ułatwia łączenie się fazy wodnej i olejowej. Substancja konserwująca, która uniemożliwia rozwój i przetrwanie mikroorganizmów w czasie przechowywania produktu. Chroni również kosmetyk przed nadkażeniem bakteryjnym, które możemy wprowadzić przy codziennym użytkowaniu produktu. Składnik dozwolony do stosowania w kosmetykach w ograniczonym stężeniu. Znajduje się na liście substancji konserwujących dozwolonych do stosowania z ograniczeniami w produktach kosmetycznych. Jego dopuszczalne maksymalne stężenie w gotowym produkcie wynosi 0,1\% (jeżeli składnik pełni rolę substancji konserwującej).

\item \textbf{Phenoxyethanol}

Fenoksyetanol. Substancja konserwująca, która uniemożliwia rozwój i przetrwanie mikroorganizmów w czasie przechowywania produktu. Chroni również kosmetyk przed zakażeniem bakteryjnym, które możemy wprowadzić przy codziennym użytkowaniu produktu. Składnik dozwolony do stosowania w kosmetykach w ograniczonym stężeniu. Znajduje się na liście substancji konserwujących dozwolonych do stosowania z ograniczeniami w produktach kosmetycznych. Jego dopuszczalne maksymalne stężenie w gotowym produkcie to 1,0\%

\item \textbf{Benzoic Acid}

Kwas benzoesowy. Substancja konserwująca, która uniemożliwia rozwój i przetrwanie mikroorganizmów w czasie przechowywania produktu. Chroni również kosmetyk przed nadkażeniem bakteryjnym, które możemy wprowadzić przy codziennym użytkowaniu produktu. Kwas benzoesowy jest dozwolony do stosowania w kosmetykach w ograniczonej ilości. Znajduje się na liście substancji konserwujących dozwolonych do stosowania. Ze względu na szerokie zastosowanie tego związku w różnego rodzaju preparatach dopuszczalne jest różne jego stężęnie w zależności od rodzaju kosmetyku. Maksymalne stężenie kwasu benzoesowego i jego soli sodowej w gotowym kosmetyku wynosi:

produkty spłukiwane z wyjątkiem preparatów do jamy ustnej: do 2,5\% (w przeliczeniu na czysty kwas benzoesowy, w przypadku stosowania soli sodowej kwasu benzoesowego);

produkty do jamy ustnej: do 1,7\% (w przeliczeniu na czysty kwas benzoesowy, w przypadku stosowania soli sodowej kwasu benzoesowego);

produkty niespłukiwane: do 0,5\% (w przeliczeniu na czysty kwas benzoesowy, w przypadku stosowania soli sodowej kwasu benzoesowego);

inne pochodne we wszystkich wyrobach do 0,5\% (w przeliczniu na czysty kwas benzoesowy).

\item \textbf{Dehydroacetic Acid}

Kwas dehydrooctowy. Substancja otrzymywana naturalnie lub syntetycznie, lecz identyczna z występującym w naturze kwasem dehydrooctowym. Dopuszczalne stężenie w kosmetykach: 0,6\%. Łagodny konserwant. Zapobiega zepsuciu kosmetyku, chroni przed mikroorganizmami.
\end{itemize}
